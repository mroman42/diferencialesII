% Compilación:
%   pdflatex --shell-escape
\documentclass[11pt]{article}
\usepackage[utf8]{inputenc}
\usepackage[T1]{fontenc}
\usepackage[spanish]{babel}

% Paquetes para el uso con latex
\usepackage{fixltx2e}
\usepackage{graphicx}
\usepackage{grffile}
\usepackage{longtable}
\usepackage{wrapfig}
\usepackage{rotating}
\usepackage[normalem]{ulem}
\usepackage{amsmath}
\usepackage{textcomp}
\usepackage{amssymb}
\usepackage{capt-of}
\usepackage[colorlinks=true]{hyperref}
\usepackage{amsthm}
\usepackage{amsmath}
\usepackage{mathtools}
\usepackage{scalerel}
\usepackage{stmaryrd}
\usepackage[a4paper]{geometry}

% Estilos para los entornos matemáticos.
\theoremstyle{plain}
\newtheorem{theorem}{Teorema}
\newtheorem{proposition}{Proposición}
\newtheorem{lemma}{Lema}
\newtheorem{corollary}{Corolario}
\theoremstyle{definition}
\newtheorem{definition}{Definición}
\newtheorem{proofs}{Demostración}
\theoremstyle{remark}
\newtheorem{remark}{Nota}
\newtheorem{exampleth}{Ejemplo}
\begingroup
  \makeatletter\@for\theoremstyle:=definition,remark,plain\do{
    \expandafter\g@addto@macro\csname th@\theoremstyle\endcsname{\addtolength\thm@preskip\parskip}}
\endgroup

% Fondo de página de la Universidad de Granada.
\usepackage{wallpaper}\ThisULCornerWallPaper{1}{ugrA4.pdf}

% Fuentes seleccionadas
\usepackage{libertine}
\usepackage{inconsolata}


% Autor y detalles del documento.
\author{Mario Román}
\date{\today}
\title{Teorema de Picard-Lindelöf}

\hypersetup{
 pdfauthor={Mario Román},
 pdftitle={Teorema de Picard-Lindelöf},
 pdfkeywords={},
 pdfsubject={},
 pdfcreator={pdflatex}, 
 pdflang={Spanish}
}

\begin{document}
\maketitle


% Macros
% Restricción de funciones
\newcommand\restr[2]{{
  \left.\kern-\nulldelimiterspace#1
    \vphantom{\big|}
  \right|_{#2}
}}

% Norma y valor absoluto
\newcommand{\DeclareAutoPairedDelimiter}[3]{
  \expandafter\DeclarePairedDelimiter\csname Auto\string#1\endcsname{#2}{#3}
  \begingroup\edef\x{\endgroup
    \noexpand\DeclareRobustCommand{\noexpand#1}{
      \expandafter\noexpand\csname Auto\string#1\endcsname*}}
  \x}
\DeclareAutoPairedDelimiter{\abs}{\lvert}{\rvert}
\DeclareAutoPairedDelimiter{\norm}{\lVert}{\rVert}


\newpage
\setcounter{tocdepth}{2}
\tableofcontents
\newpage

\section{Preliminares}

\subsection{Teorema de Banach}

\begin{theorem}[del punto fijo de Banach]\label{punto-fijo-banach}
  Sea $(E,d)$ un espacio métrico completo y sea una aplicación contractiva
  $F \colon E \to E$; es decir, existe un real $\alpha \in (0,1)$ cumpliendo $d(F(x),F(y)) \leq \alpha d(x,y)$
  para cualesquiera $x,y \in E$. Entonces,

  \begin{itemize}
  \item la aplicación tiene un único punto fijo, es decir, $\exists! x^{\ast}\in E: F(x^{\ast}) = x^{\ast}$;
  \item y para cualquier $x_0 \in E$, las sucesivas aplicaciones de la función crean
    una sucesión que tiende al punto fijo, $\left\{ F \circ \overset{n}{\dots} \circ F (x_0) \right\}_{n \in \mathbb{N}} \longrightarrow x^{\ast}$.
  \end{itemize}

  En resumen, toda aplicación contractiva en un espacio métrico
  completo tiene un punto fijo y las aplicaciones sucesivas de la
  misma sobre cualquier punto convergen a él.
\end{theorem}


\subsection{Bases de entornos para Picard-Lindelöf}\label{base-entornos}

Dados \(a,b\in \mathbb{R}^+\), definimos la siguiente base de entornos
cerrados, que pueden interpretarse como la evolución en el tiempo de una bola
cerrada de radio \(b\).
\begin{align*}
  {\cal R}_{a,b}(t_0,x_0)
  &=
    \left\{ (t,x) \in \mathbb{R}\times \mathbb{R}^d \mid |t - t_0| < a, \|x - x_0\| \leq b \right\}
  \\&=
    [t_0-a,t_0+a] \times \overline{B}(x_0,b).
\end{align*}
El espacio que forman las funciones \(\varphi \colon [t_0 - a, t_0 + a] \to \mathbb{R}^d\) cuya gráfica queda
contenida en \({\cal R}_{a,b}\) es precisamente la bola cerrada de radio \(b\) centrada
en la función constante \(\varphi_0(t) = x_0\), una vez que consideramos la norma
del supremo.
\begin{align*}
  E_{a,b}(t_0,x_0) &=
  \left\{ \varphi \in {\cal C}([t_0-a,t_0+a],\mathbb{R}^d) \;\middle|\; 
    (t,\varphi(t)) \in {\cal R}_{a,b}(t_0,x_0)  \right\}
  \\&= \left\{ \varphi \in {\cal C}([t_0-a,t_0+a],\mathbb{R}^d) \;\middle|\;
    \forall t \in [t_0-a,t_0+a]: \|\varphi(t) - \varphi_0(t)\| \leq b \right\}
  \\&= 
  \overline{B}_{\infty}(\varphi_0,b).
\end{align*}

Al ser un subconjunto cerrado del espacio completo de funciones continuas
acotadas con la norma del supremo, es un espacio métrico completo
cuando consideramos la distancia inducida por la norma, \(d_{\infty}(\varphi,\psi) = \|\varphi - \psi\|_{\infty}\).

Usaremos durante esta exposición las abreviaturas \(E_{a,b} = E_{a,b}(t_0,x_0)\)
y \({\cal R}_{a,b} = {\cal R}_{a,b}(t_0,x_0)\).

% TODO: Probar que forma una base de entornos.
% TODO: Comprobar la norma del supremo sobre vectores y complitud.

\subsection{Operador integral de Volterra}

Sea un problema de valores iniciales para el campo $f \colon D \to \mathbb{R}^d$ con $D \subseteq \mathbb{R}\times\mathbb{R}^d$
abierto, dado por
\[\left.\begin{array}{l}
          x' = f(t,x), \\
          x(t_0) = x_0.
\end{array}\right\}\]
Usando que $D$ es abierto, para $a,b$ suficientemente pequeños tendremos ${\cal R}_{a,b}(t_0,x_0) \subset D$,
y podremos definir el \textbf{operador de Volterra} asociado al problema en el
espacio de funciones $V \colon E_{a,b} \to {\cal C}([t_0-a,t_0+a], \mathbb{R}^d)$, como
\[
  V(\varphi)(t) = x_0 + \int_{t_0}^t f(s,\varphi(s))\;ds.
\]
Nótese que un punto fijo $V(\varphi) = \varphi$ del operador sería una solución al problema
de Volterra asociado al PVI original en el intervalo abierto $(t_0-a,t_0+a)$, lo
que equivaldría a ser solución del PVI en ese intervalo. Lo que haremos será
buscar las condiciones en las que $V$ es un operador contractivo al que se
puede aplicar el Teorema del punto fijo de Banach, y usarlo para asegurar
la existencia y unicidad de solución del PVI\@. Empezaremos acotando el
operador de Volterra desde una acotación de $f$. En particular, obtendremos $V(E) \subset E$
cuando el valor de $a$ sea suficientemente pequeño.


\begin{lemma}[sobre el operador de Volterra]\label{lema-operador-volterra}
  Si \(M\geq 0\) cumple que \(\forall (t,x) \in {\cal R}_{a,b}: \|f(t,x)\| \leq M\), entonces
  \(\|V(\varphi) - \varphi_0\|_{\infty} \leq Ma\) para cualquier \(\varphi \in E_{a,b}\).
\end{lemma}
\begin{proof}
  Sabiendo que \(\varphi \in E_{a,b}\), tenemos para cualquier \(s \in [t_0-a,t_0+a]\) que \((s,\varphi(s)) \in {\cal R}_{a,b}\);
  por lo que \(\|f(s,\varphi(s))\| \leq M\) y se tiene la siguiente cadena de desigualdades.
  \[ 
    \| V(\varphi) - \varphi_0 \|_{\infty} \leq
    \left\| \int_{t_0}^t f(s,\varphi(s))\,ds \right\|_{\infty} \leq
    M\| t - t_0 \|_{\infty} \leq
    Ma.\qedhere
  \]
\end{proof}

\begin{corollary}[]\label{corolario-1-volterra}
   Si \(M \geq 0\) cumple que \(\forall (t,x) \in {\cal R}_{a,b}: \|f(t,x)\| \leq M\) y además
   \(Ma \leq b\), se tiene \(V(E_{a,b}) \subset E_{a,b}\).
\end{corollary}
\begin{proof}
  Nótese que si leemos \(E_{a,b}\) como la bola cerrada \(\overline{B}_{\infty}(\varphi_0,b)\), tenemos
  por el Lema~\ref{lema-operador-volterra} que \(V(E) \subset \overline{B}_{\infty}(\varphi_0,Ma) \subset \overline{B}_{\infty}(\varphi_0,b)\).
\end{proof}



Por otro lado, podremos probar que todas las soluciones de la ecuación
de Volterra están contenidas en el espacio $E_{a,b}$.

\begin{lemma}[]\label{lema-2-contr}
  Para \(M \geq 0\) cumpliendo que \(\forall (t,x)\in {\cal R}_{a,b} : \|f(t,x)\| \leq M\) y además
  \(Ma \leq b\); se tiene que si \(\varphi \colon [t_0-a,t_0+a] \to \mathbb{R}^d\) es solución
  de la ecuación de Volterra, entonces \(\varphi \in E_{a,b}\).
\end{lemma}
\begin{proof}
  Afirmamos que $\|\varphi-x_0\|_{\infty} \leq b$ en $[t_0-a,t_0+a]$. Sabemos que $\|\varphi(t_0) - x_0\| = 0$, luego
  si no fuera cierto para algún $t_1 > t_0$ en el intervalo, por el \textbf{Lema del primer instante}
  existiría un $\tau \in (t_0,t_0+a]$ tal que $b<\|\varphi(\tau) - x_0\|$ que sería el primero,
  es decir, $\forall t \in [t_0,\tau)\colon \|\varphi(t)-x_0\| \leq b$. Pero entonces, $\forall t \in [t_0,\tau)\colon (t,\varphi(t)) \in {\cal R}_{a,b}$ y
  \[\begin{aligned}
   b < \|\varphi(\tau)-x_0\| &=
   \left\| \int_{t_0}^t f(s,\varphi(s))\,ds \right\| \leq
   \int_{t_0}^t \left\| f(s,\varphi(s)) \right\|\,ds \\&\leq
   M|t-t_0| \leq
   Ma \leq b,
 \end{aligned}\]
  llegando a contradicción. Del mismo modo, si no fuera cierto para algún $t_1 < t_0$,
  aplicaríamos análogamente el lema del primer instante para obtener un $\tau \in [t_0-a,t_0)$
  con $b < \| \varphi(\tau) - x_0 \|$ que cumpliría $\forall t \in (\tau,t_0]\colon \|\varphi(t)-x_0\| \leq b$. Pero entonces, de
  nuevo,  
  \[\begin{aligned}
   b < \|\varphi(\tau)-x_0\| &=
   \left\| -\int_{t}^{t_0} f(s,\varphi(s))\,ds \right\| \leq
   \int^{t_0}_t \left\| f(s,\varphi(s)) \right\|\,ds \\&\leq
   M|t-t_0| \leq
   Ma \leq b,
 \end{aligned}\]
  y habríamos llegado igualmente a contradicción.
\end{proof}

Lo que probamos a continuación es que una condición de lipschitzianidad respecto
de la segunda variable para el campo que determina la ecuación diferencial nos
da una condición de lipschitzianidad sobre el operador de Volterra.

\begin{lemma}[]\label{lema-3-contr}
  Para $L \geq 0$ tal que
  \[
    \forall (t,x),(t,y) \in {\cal R}_{a,b}\colon
    \| f(t,x) - f(t,y) \| \leq L \|x-y\|, 
  \]
  se tiene $\|V(\varphi) - V(\psi)\|_\infty \leq aL\|\varphi - \psi\|_\infty$ para cualesquiera $\varphi,\psi \in E_{a,b}$.
\end{lemma}
\begin{proof}
  Para $t \in [t_0-a,t_0+a]$, si suponemos $t > t_0$ tenemos
  \[\begin{aligned}
  \| V(\varphi)(t) - V(\psi)(t) \| &=
  \left\| \int_{t_0}^t \left( f(s,\varphi(s)) - f(s,\psi(s)) \right)\,ds \right\| \\&\leq
  \int_{t_0}^t \left\|  f(s,\varphi(s)) - f(s,\psi(s)) \right\|\,ds \\&\leq
  \int_{t_0}^t L\left\| \varphi(s)-\psi(s) \right\|\,ds \\&\leq
  La\|\varphi-\psi\|_{\infty}.
  \end{aligned}\]
En el caso $t = t_0$, tendremos $\|V(\varphi)(t) - V(\psi)(t)\| = 0$;
y en el caso $t < t_0$, realizaremos un razonamiento análogo al anterior
pero ahora notando que
\[\begin{aligned}
  \left\| - \int_{t_0}^t  (f(s,\varphi(s)) - f(s,\psi(s))) \,ds \right\| \leq
  \int_{t}^{t_0} \left\|  f(s,\varphi(s)) - f(s,\psi(s)) \right\|\,ds.
\end{aligned}\]
Así, hemos acotado la diferencia en todos los casos y por tanto hemos
acotado la norma del supremo.
\end{proof}

Tomando así un $a$ suficientemente pequeño, podremos obtener una primera
aplicación del Teorema de Banach a nuestro caso particular.

\begin{proposition}\label{proposicion-1-contr}
Sean $M,L \geq 0$ tales que
\[
\forall (t,x) \in R_{a,b}\colon \|f(t,x)\| \leq M,
\]
y que
\[
\forall (t,x),(t,y) \in {\cal R}_{a,b}\colon \|f(t,x)-f(t,y)\| \leq L\|x-y\|.
\]
Si $aM\leq b$ y $aL \leq 1$, entonces existe una única $\varphi\colon [t_0-a,t_0+a] \to \mathbb{R}^d$
solución al problema de Volterra.
\end{proposition}
\begin{proof}
Aplicando el Corolario~\ref{corolario-1-volterra} a $M$, tenemos $V(E_{a,b}) \subset E_{a,b}$;
y aplicando el Lema~\ref{lema-3-contr} a $L$, obtenemos
\[
\| V(\varphi) - V(\psi) \|_{\infty} \leq aL\|\varphi-\psi\|_{\infty}.
\]
Ahora, como $aL < 1$, por el Teorema del punto fijo de Banach, $\exists! \varphi \in E_{a,b} \colon V(\varphi) = \varphi$,
solución de Volterra. Si hubiera otra solución $\psi$ en mismo intervalo usaríamos
el Lema~\ref{lema-2-contr} para tener $\psi \in E_{a,b}$, y como el Teorema de Banach nos
da unicidad en ese espacio, $\varphi = \psi$.
\end{proof}

Nuestro siguiente objetivo será retirar la hipótesis $aL < 1$, para lo que
utilizaremos la norma de Bielecki.

\section{Norma de Bielecki}

La \textbf{norma de Bielecki} para una constante \(R > 0\) se define en el
espacio \(E_{a,b}\) como
\[\|\varphi\|_B = \max_{t \in [t_0-a,t_0+a]} e^{-R|t-t_0|}\|\varphi(t)\|.\]

\begin{remark}[de equivalencia con la norma del máximo]
  La norma de Bielecki es equivalente a la norma del máximo. Explícitamente,
  se tiene que
  \[
    \|\varphi\|_B = \max_{t \in [t_0-a,t_0+a]} e^{-R|t-t_0|} \|\varphi(t)\| \leq
     e^{-Ra} \|\varphi\|_\infty
   \]
   y que
   \[
     \|\varphi\|_\infty =  \max_{t \in [t_0-a,t_0+a]} e^{R|t-t_0|}e^{-R|t-t_0|} \|\varphi(t)\| \leq
     e^{Ra}\|\varphi\|_B.
     \]
\end{remark}

\begin{lemma}[versión del lema 3 con la norma de Bielecki]\label{lema-4-contr}
  Para \(L \geq 0\) tal que
  \[
    \forall (t,x),(t,y) \in {\cal R}_{a,b} : \| f(t,x) - f(t,y) \| \leq L\|x-y\|,
  \]
  se tiene \(\|V(\varphi)-V(\psi)\|_B \leq \frac{L}{R} \|\varphi - \psi\|_B\) para cualesquiera $\varphi,\psi \in E_{a,b}$.
\end{lemma}
\begin{proof}
Nótese primero que $\|V(\varphi)(t_0) - V(\psi)(t_0)\| = 0$. Fijado un $t > t_0$
usamos la cota de $L$ y la definición de la norma de Bielecki para tener
\[\begin{aligned}
\norm{V(\varphi)(t) - V(\psi)(t)} &=
\norm{\int_{t_0}^t f(s,\varphi(s)) - f(s,\psi(s))\,ds} \leq
\int_{t_0}^t \norm{f(s,\varphi(s)) - f(s,\psi(s))}\,ds \\&\leq
L \int_{t_0}^t \norm{\varphi(s) - \psi(s)}\,ds =
L \int_{t_0}^t \norm{\varphi(s) - \psi(s)} e^{-R\abs{s-t_0}}e^{R\abs{s-t_0}}\, ds \\&\leq
L \norm{\varphi - \psi}_{B} \int_{t_0}^t e^{R\abs{s-t_0}}\,ds \leq
\frac{L}{R} \norm{\varphi - \psi}_{B} \left(e^{R\abs{t-t_0}} - 1\right) \\&\leq
\frac{L}{R} \norm{\varphi - \psi}_{B} e^{R\abs{t-t_0}}
\end{aligned}\]
ya que $e^{R|t-t_0|} > 0$. Si fijamos un $t < t_0$ podemos repetir el mismo razonamiento
simplemente notando al principio que
\[
\norm{- \int_{t_0}^t f(s,\varphi(s)) - f(s,\psi(s))\,ds} \leq
\int_{t}^{t_0} \norm{f(s,\varphi(s)) - f(s,\psi(s))}\,ds.
\]
Dividiendo y uniendo los tres casos, llegamos a que
$\|V(\varphi) - V(\psi)\|_{B} \leq \frac{L}{R} \norm{\varphi - \psi}_{B}$.
\end{proof}

\begin{proposition}\label{proposicion-2-contr}
Sean $M,L \geq 0$ tales que
\[
\forall (t,x) \in R_{a,b}\colon \|f(t,x)\| \leq M,
\]
y que
\[
\forall (t,x),(t,y) \in {\cal R}_{a,b}\colon \|f(t,x)-f(t,y)\| \leq L\|x-y\|.
\]
Si $aM\leq b$, entonces existe una única $\varphi\colon [t_0-a,t_0+a] \to \mathbb{R}^d$
solución al problema de Volterra.
\end{proposition}
\begin{proof}
  Si tomamos $R > L$, podemos aplicar el Lema~\ref{lema-4-contr} para
  obtener que $V$ es contractiva en el espacio métrico $E_{a,b}$ con la
  norma de Bielecki. Repitiendo el razonamiento de la Proposición~\ref{proposicion-1-contr},
  obtenemos que $V(E_{a,b}) \subset E_{a,b}$ y que existirá un punto
  fijo de $V$.
\end{proof}

\section{Teorema de Picard-Lindelöf}

\subsection{Teorema de Picard-Lindelöf}

\begin{theorem}[Picard-Lindelöf local]\label{picard-lindelof-local}
  Dado el siguiente problema de valores iniciales con \(D \subset \mathbb{R}\times \mathbb{R}^d\) dominio,
  \((t_0,x_0) \in D\) y \(f \colon D \to \mathbb{R}^N\) continua,
  \[\left\{\begin{array}{l}
             x' = f(t,x), \\
             x(t_0) = x_0;
           \end{array}\right.\]
  si \(f\) es localmente lipschitziana respecto de \(x\) en un entorno ${\cal U}$ de \((t_0,x_0)\),
  entonces existe solución al problema de valores iniciales y es
  localmente única.
\end{theorem}
\begin{proof}
  Por hipótesis sabemos que existe una constante de Lipschitz $L \in \mathbb{R}^+$
  tal que
  \[ \forall (t,x),(t,y) \in {\cal U}\colon \| f(t,x) - f(t,y) \| \leq L\|x-y\|. \]
  Usando que la familia de entornos ${\cal R}_{a,b}$ definida en la Sección~\ref{base-entornos}
  forma una base de entornos, sabemos que existen $\overline{a},b \in \mathbb{R}^+$ tales que ${\cal R}_{\overline{a},b}(t_0,x_0) \subseteq {\cal U}$.
  Ahora, $\restr{f}{{\cal R}_{\overline{a},b}}$ será globalmente lipschitziana, luego continua, y podremos
  aplicar el Teorema de Weierstrass para obtener la cota
  \[M = \max_{(t,x) \in {\cal R}_{\overline{a},b}} \|f(t,x)\|.\]
  Definimos ahora $a = \min\left\{ \overline{a}, b/M, 1/L \right\}$ y, sabiendo que ${\cal R}_{a,b} \subseteq {\cal R}_{\overline{a},b}$, podemos aplicar
  la Proposición~\ref{proposicion-2-contr} para obtener que existe una solución
  localmente única $\varphi \colon [t_0-a,t_0+a] \to \mathbb{R}^d$ a la ecuación de Volterra, que en el
  abierto $\varphi \colon I = (t_0-a,t_0+a) \to \mathbb{R}^d$ será solución del PVI\@. Nótese sin embargo que
  todavía no hemos probado la unicidad local.
  
  Sea ahora $\psi \colon J \to \mathbb{R}^d$ otra solución en un intervalo abierto conteniendo
  a $t_0$; por continuidad existe un $\delta' < a$ tal que para $|t - t_0| < \delta'$ se tiene
  $\|\psi(t) - x_0\| \leq b$. Como ambos intervalos son abiertos y $t_0 \in I \cap J$, existirá
  un entorno de $t_0$ contenido en la intersección y podremos tomar un
  $\delta < \delta'$ suficientemente pequeño para asegurar que $(t_0-\delta, t_0+\delta) \subset I \cap J$.
  Tendríamos así dos soluciones al problema de Volterra $\psi,\varphi \in E_{\delta,b}(t_0,x_0)$,
  donde además sabemos que $\delta M < aM < b$, luego, de nuevo por la
  Proposición~\ref{proposicion-2-contr}, coincidirán en $[t_0-\delta,t_0+\delta]$.
\end{proof}


\begin{theorem}[Picard-Lindelöf global]\label{picard-lindelof-global}
  Dado el siguiente problema de valores iniciales con \(D \subset \mathbb{R} \times \mathbb{R}^d\) dominio,
  \((t_0,x_0) \in D\) y \(f \colon D \to \mathbb{R}^N\) continua,
  \[\left\{\begin{array}{l}
                  x' = f(t,x), \\
                  x(t_0) = x_0;
                \end{array}\right.\] 
  si \(f\) es localmente lipschitziana respecto de \(x\), entonces existe solución
  al problema de valores iniciales y es la única solución maximal.
\end{theorem}
\begin{proof}
  Cualquier $(t_0,x_0)$ tiene un entorno (el dominio) donde podemos aplicar
  la versión local del Teorema~\ref{picard-lindelof-local}. Finalmente, sabemos
  que la propiedad de unicidad local bajo cualesquiera condiciones iniciales
  implica la unicidad global bajo cualquier condición inicial.
\end{proof}



\subsection{Iterantes de Picard}

La segunda consecuencia que obtenemos del Teorema de Banach es la posibilidad
de aproximar la solución del PVI mediante la aplicación sucesiva del operador
de Volterra en los casos en los que hemos comprobado que es contractivo.

\begin{theorem}[Iterantes de Picard]
  Supongamos que tenemos constantes \(a,b,M,L > 0\) y un dominio \(D \subset \mathbb{R} \times \mathbb{R}^d\)
  cumpliendo
  \begin{itemize}
    \item \(a \leq b/M\),
    \item \({\cal R}_{a,b} \subset D\),
    \item \(\forall (t,x) \in {\cal R}_{a,b}\colon \|f(t,x)\| \leq M\),
    \item \(\forall (t,x),(t,y) \in {\cal R}_{a,b} \colon \|f(t,x)-f(t,y)\| \leq L \|x-y\|\).
  \end{itemize}
  Entonces la sucesión de funciones definida recursivamente mediante aplicaciones
  sucesivas del operador integral de Volterra
  \begin{itemize}
    \item \(\phi_0(t) = x_0\),
    \item \(\phi_{k+1}(t) = x_0 + \int_{t_0}^t f(s,\phi_k(s))\,ds\),
  \end{itemize}
  está bien definida y converge uniformemente hacia la única solución
  de la ecuación integral de Volterra. Esta sucesión, \(\left\{ \phi_n \right\}\), se conoce
  como sucesión de \textbf{iterantes de Picard}.
\end{theorem}

\subsection{Contraejemplo de Müller}

El \textbf{contraejemplo de Müller} (1927) considera el PVI
\[\left\{\begin{array}{l}
  x' = f(t,x) \\
  x(0) = 0
\end{array}\right.\]
para \(f \colon \mathbb{R} \times \mathbb{R} \to \mathbb{R}\) una función definida a trozos como
\[ f(t,x) = \left\{
    \begin{array}{ll}
      0 & \mbox{ si } t \leq 0, \\
      2t & \mbox{ si } t > 0 \mbox{ y } x < 0, \\
      2t - \frac{4x}{t} & \mbox{ si } t > 0 \mbox{ y } 0 \leq x < t^2, \\
      -2t & \mbox{ si } t > 0 \mbox{ y } x > t^2. \\
    \end{array}\right.\]
Este es un PVI con una solución única a la que no se aproximará
la sucesión de iterantes de Picard, ni siquiera en ninguna de sus
parciales.

Podemos comprobar que tiene una única solución aplicando el Teorema
de Unicidad de Peano. En efecto, la unicidad en el futuro se obtiene
del hecho de que para cualquier $t>0$, la función
\[ f(t,x) = \left\{
    \begin{array}{ll}
      2t & \mbox{ si } x < 0, \\
      2t - \frac{4x}{t} & \mbox{ si } 0 \leq x < t^2, \\
      -2t & \mbox{ si } x > t^2, \\
    \end{array}\right.\]
está formada por tres trozos de recta y es decreciente en $x$. Del
mismo modo, la unicidad en el pasado se obtiene del hecho de que
para $t < 0$, se tiene $f(t,x) = 0$ y por tanto es creciente. De
la conjunción de las propiedades de unicidad en el futuro y en el
pasado obtenemos la unicidad global de la solución, si existiera.

Además, podemos calcular explícitamente la solución. Si buscamos
soluciones a la ecuación lineal
\[
x' = 2t - \frac{4x}{t};
\]
obtendremos posibles soluciones de la forma $x = t^2/3 + A/t^4$. En
nuestro caso, tendremos que tomar $A = 0$ para que pueda cumplirse
la condición de valores iniciales y que esté bien definida en $0$.
Concatenaremos con una solución constantemente $0$ antes de la
condición inicial para comprobar que, efectivamente,
\[
  \varphi(t) = \left\{\begin{array}{ll}
      0 &\mbox{ si } t < 0 \\
      t^2/3 &\mbox{ si } t \geq 0   
  \end{array}\right.
\]
es continua y derivable y es solución de nuestro problema de
valores iniciales, ya que para $t > 0$ se tiene $f(t,t^2/3) = 2t - 4/3t = 2/3t = \varphi'(t)$,
donde usamos crucialmente que $t^2/3 < t^2$.

Sin embargo, si intentamos aproximarla por iterantes de Picard
obtendríamos,
\begin{itemize}
\item $\phi_0(t) = 0$,
\item $\displaystyle\phi_1(t) = \int_0^t f(s,0)\,ds = \int_0^t 2s\,ds = t^2$,
\item $\displaystyle\phi_2(t) = \int_0^t f(s,s^2)\,ds = \int_0^t -2s\,ds = -t^2$,
\item $\displaystyle\phi_3(t) = \int_0^t f(s,-s^2)\,ds = \int_0^t 2s\,ds = t^2$, $\dots$
\end{itemize}
y sucesivamente iteraríamos entre $t^2$ y $-t^2$, que no se acercan
a la solución ni siquiera en ninguna de sus parciales.


\section{Teorema de Cauchy-Peano}

\bibliographystyle{alpha}
\bibliography{Bibliography}
\end{document}
