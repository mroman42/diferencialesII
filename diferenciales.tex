% Compilación:
%   pdflatex --shell-escape
%   bibtex
%   pdflatex --shell-escape
%   pdflatex --shell-escape

\documentclass[11pt]{article}
\usepackage[utf8]{inputenc}
\usepackage[T1]{fontenc}
\usepackage[spanish]{babel}
\usepackage[toc,page]{appendix}
\addto\captionsspanish{
  \renewcommand\appendixname{Anexo}
  \renewcommand\appendixpagename{Anexos}
  \renewcommand\appendixtocname{Anexos}
}

% Paquetes para el uso con latex
\usepackage{parskip}
\usepackage{fixltx2e}
\usepackage{graphicx}
\usepackage{grffile}
\usepackage{longtable}
\usepackage{wrapfig}
\usepackage{rotating}
\usepackage[normalem]{ulem}
\usepackage{amsmath}
\usepackage{textcomp}
\usepackage{amssymb}
\usepackage{capt-of}
\usepackage[colorlinks=true,linkcolor=ugrRed,citecolor=ugrRed]{hyperref}
\usepackage{amsthm}
\usepackage{amsmath}
\usepackage{mathtools}
\usepackage{scalerel}
\usepackage{stmaryrd}
\usepackage[a4paper]{geometry}
\usepackage{tikz}
\usetikzlibrary{shapes}
\usepackage{pgfplots}

% Estilos para los entornos matemáticos.
\theoremstyle{plain}
\newtheorem{theorem}{Teorema}
\newtheorem{proposition}{Proposición}
\newtheorem{lemma}{Lema}
\newtheorem{corollary}{Corolario}
\theoremstyle{definition}
\newtheorem{definition}{Definición}
\newtheorem{proofs}{Demostración}
\theoremstyle{remark}
\newtheorem{remark}{Nota}
\newtheorem{exampleth}{Ejemplo}
\begingroup
  \makeatletter\@for\theoremstyle:=definition,remark,plain\do{
    \expandafter\g@addto@macro\csname th@\theoremstyle\endcsname{\addtolength\thm@preskip\parskip}}
\endgroup

% Fondo de página de la Universidad de Granada.
\usepackage{wallpaper}\ThisULCornerWallPaper{1}{ugrA4.pdf}

% Fuentes seleccionadas
\usepackage{libertine}
\usepackage{inconsolata}
\definecolor{ugrRed}{HTML}{c6474b}

% Autor y detalles del documento.
\author{Mario Román}
\date{\today}
\title{Teorema de Picard-Lindelöf y Teorema de Cauchy-Peano}

\hypersetup{
 pdfauthor={Mario Román},
 pdftitle={Teorema de Picard-Lindelöf},
 pdfkeywords={},
 pdfsubject={},
 pdfcreator={pdflatex}, 
 pdflang={Spanish}
}

\begin{document}
\maketitle



%%----------------------------------------------------------------
%%      Macros, operadores y símbolos
%%----------------------------------------------------------------

\newcommand\restr[2]{#1_{|_{#2}}}

% Norma, valor absoluto y otros delimitadores
\newcommand{\DeclareAutoPairedDelimiter}[3]{
  \expandafter\DeclarePairedDelimiter\csname Auto\string#1\endcsname{#2}{#3}
  \begingroup\edef\x{\endgroup
    \noexpand\DeclareRobustCommand{\noexpand#1}{
      \expandafter\noexpand\csname Auto\string#1\endcsname*}}
  \x}
\DeclareAutoPairedDelimiter{\abs}{\lvert}{\rvert}
\DeclareAutoPairedDelimiter{\norm}{\lVert}{\rVert}
\DeclareAutoPairedDelimiter{\set}{\{}{\}}
\newcommand{\midd}{\;\middle|\;}

% Operadores
\def\spanishoperators{adj traza vect dom dist sop vol sgn  Hess Jac rango grado diag img longitud Maximizar Minimizar Optimizar sec cotan cosec}
\newcommand{\w}{\displaystyle}
\newcommand{\pre}[1]{\left\langle#1\right\rangle}


% Símbolos
\newcommand{\RR}{\mathbb{R}}
\newcommand{\NN}{\mathbb{N}}
\newcommand{\ZZ}{\mathbb{Z}}
\newcommand{\eR}{{\cal{R}}}
\newcommand{\oB}{\overline{B}}
\newcommand{\tf}{\widetilde{f}}


\newpage
\setcounter{tocdepth}{2}
\tableofcontents
\newpage

\section{Preliminares}

\subsection{Teorema de Banach}

\begin{theorem}[del punto fijo de Banach]\label{punto-fijo-banach}
  Sea $(E,d)$ un espacio métrico completo y sea una aplicación contractiva
  $F \colon E \to E$; es decir, existe un real $\alpha \in (0,1)$ cumpliendo $d(F(x),F(y)) \leq \alpha d(x,y)$
  para cualesquiera $x,y \in E$. Entonces,

  \begin{itemize}
  \item la aplicación tiene un único punto fijo, es decir, $\exists! x^{\ast}\in E: F(x^{\ast}) = x^{\ast}$;
  \item y para cualquier $x_0 \in E$, las sucesivas aplicaciones de la función crean
    una sucesión que tiende al punto fijo, $\set{F \circ \overset{n}{\dots} \circ F (x_0)}_{n \in \mathbb{N}} \longrightarrow x^{\ast}$.
  \end{itemize}

  En resumen, toda aplicación contractiva en un espacio métrico
  completo tiene un punto fijo y las aplicaciones sucesivas de la
  misma sobre cualquier punto convergen a él.
\end{theorem}

\subsection{Problema de valores iniciales}

Salvo que se especifique lo contrario, trabajaremos siempre en el contexto del
siguiente problema de valores iniciales (PVI).  Fijado el abierto $D \subseteq \mathbb{R} \times \mathbb{R}^d$,
consideramos el campo de vectores $f \colon D \to \RR^d$ continuo y un valor inicial $(t_0,x_0)$.
El problema viene dado como
\[\left.\begin{array}{l}
          x' = f(t,x), \\
          x(t_0) = x_0.
\end{array}\right\}\]

\subsection{Bases de entornos para Picard-Lindelöf}\label{base-entornos}

Dados $a,b\in \RR^+$, y fijada una norma cualquiera en $\RR^d$,
definimos la siguiente base de entornos cerrados, que pueden interpretarse
como la evolución en el tiempo de una bola cerrada de radio $b$. Por ser
cerrado y acotado del espacio euclídeo, será un compacto.
\begin{align*}
  \eR_{a,b}(t_0,x_0)
  &=  \set{(t,x) \in \RR\times \RR^d \midd  |t - t_0| < a, \|x - x_0\| \leq b}
  \\&=  [t_0-a,t_0+a] \times \oB(x_0,b).
\end{align*}
El espacio que forman las funciones continuas \(\varphi \colon [t_0 - a, t_0 + a] \to \RR^d\) cuya gráfica queda
contenida en ${\eR}_{a,b}$ es precisamente la bola cerrada de radio \(b\) centrada
en la función constante $\varphi_0(t) = x_0$ una vez que consideramos la norma
del supremo sobre la norma que habíamos fijado en $\RR^d$.
\begin{align*}
  E_{a,b}(t_0,x_0) &=
  \set{\varphi \in {\cal C}([t_0-a,t_0+a],\RR^d) \midd
    (t,\varphi(t)) \in {\eR}_{a,b}(t_0,x_0)}
  \\&= \set{\varphi \in {\cal C}([t_0-a,t_0+a],\RR^d) \midd
    \forall t \in [t_0-a,t_0+a]: \|\varphi(t) - \varphi_0(t)\| \leq b}
  \\&=  
  \overline{B}_{\infty}(\varphi_0,b).
\end{align*}

Al ser un subconjunto cerrado del espacio completo de funciones continuas
acotadas con la norma del supremo, que forman un espacio de Banach, es un espacio métrico
completo con la distancia inducida por la norma del máximo,
\(d_{\infty}(\varphi,\psi) = \|\varphi - \psi\|_{\infty}\).

Usaremos durante esta exposición las abreviaturas $E_{a,b} = E_{a,b}(t_0,x_0)$
y ${\eR}_{a,b} = {\eR}_{a,b}(t_0,x_0)$.


\subsection{Operador integral de Volterra}

Usando que $D$ es abierto y $(t_0,x_0) \in D$, para cualesquiera $a,b \in \RR^+$ suficientemente pequeños
tendremos ${\eR}_{a,b} \subset D$,
y podremos definir el \textbf{operador de Volterra} asociado al problema en el
espacio de funciones $V \colon E_{a,b} \to {\cal C}([t_0-a,t_0+a], \RR^d)$, como
\[ V(\varphi)(t) = x_0 + \int_{t_0}^t f(s,\varphi(s))\;ds. \]
Nótese que un punto fijo $V(\varphi) = \varphi$ del operador daría una solución al problema
de Volterra asociado al PVI original en $[t_0-a,t_0+a]$ y particular en el intervalo abierto $(t_0-a,t_0+a)$, lo
que equivaldría a ser solución del PVI en ese intervalo. Lo que haremos será
buscar las condiciones en las que $V$ es un operador contractivo al que se
puede aplicar el Teorema del punto fijo de Banach, y usarlo para asegurar
la existencia de solución del PVI\@. Empezaremos acotando el
operador de Volterra desde una acotación de $f$. En particular, obtendremos $V(E) \subset E$
cuando el valor de $a$ sea suficientemente pequeño.


\begin{lemma}[sobre el operador de Volterra]\label{lema-operador-volterra}
  Si $M\geq 0$ cumple que $\forall (t,x) \in \eR_{a,b}\colon \norm{f(t,x)} \leq M$, entonces
  \(\|V(\varphi) - \varphi_0\|_{\infty} \leq Ma\) para cualquier \(\varphi \in E_{a,b}\),
  donde $\varphi_0(t) = x_0$ es la función constantemente $x_0$.
\end{lemma}
\begin{proof}
  Sabiendo que \(\varphi \in E_{a,b}\), tenemos para cualquier \(s \in [t_0-a,t_0+a]\)
  que \((s,\varphi(s)) \in {\eR}_{a,b}\); por lo que \(\|f(s,\varphi(s))\| \leq M\) y
  se tiene la siguiente cadena de desigualdades.
  \[ \| V(\varphi) - \varphi_0 \|_{\infty} \leq
    \norm{ \int_{t_0}^t f(s,\varphi(s))\,ds } \leq
    M\abs{t - t_0} \leq
    Ma.\qedhere\]
\end{proof}

\begin{corollary}[]\label{corolario-1-volterra}
  Si \(M \geq 0\) cumple que \(\forall (t,x) \in {\eR}_{a,b}: \|f(t,x)\| \leq M\)
  y además \(Ma \leq b\), se tiene \(V(E_{a,b}) \subset E_{a,b}\).
\end{corollary}
\begin{proof}
  Nótese que si leemos \(E_{a,b}\) como la bola cerrada \(\overline{B}_{\infty}(\varphi_0,b)\), tenemos
  por el Lema~\ref{lema-operador-volterra} que \(V(E) \subset \overline{B}_{\infty}(\varphi_0,Ma) \subset \overline{B}_{\infty}(\varphi_0,b)\).
\end{proof}


Por otro lado, podremos probar que, en estas condiciones,
todas las soluciones de la ecuación de Volterra están contenidas
en el espacio $E_{a,b}$.

\begin{lemma}[]\label{lema-2-contr}
  Para \(M \geq 0\) cumpliendo que \(\forall (t,x)\in {\eR}_{a,b} : \|f(t,x)\| \leq M\) y además
  \(Ma \leq b\); se tiene que si \(\varphi \colon [t_0-a,t_0+a] \to \RR^d\) es solución
  de la ecuación de Volterra, entonces \(\varphi \in E_{a,b}\).
\end{lemma}
\begin{proof}
  Afirmamos que $\|\varphi-x_0\|_{\infty} \leq b$ en $[t_0-a,t_0+a]$. Sabemos que $\|\varphi(t_0) - x_0\| = 0$, luego
  si no fuera cierto para algún $t_1 > t_0$ en el intervalo, por el \textit{Lema del primer instante},
  existiría un $\tau \in (t_0,t_0+a]$ tal que $b<\|\varphi(\tau) - x_0\|$ que sería el primero,
  es decir, $\forall t \in [t_0,\tau)\colon \|\varphi(t)-x_0\| \leq b$. Pero entonces, $\forall t \in [t_0,\tau)\colon (t,\varphi(t)) \in {\eR}_{a,b}$ y
  \[\begin{aligned}
   b < \|\varphi(\tau)-x_0\| &=
   \norm{\int_{t_0}^\tau f(s,\varphi(s))\,ds} \leq
   \int_{t_0}^\tau \norm{f(s,\varphi(s))} \,ds \\&\leq
   M\abs{\tau-t_0} \leq
   Ma \leq b,
 \end{aligned}\]
  llegando a contradicción. Del mismo modo, si no fuera cierto para algún $t_1 < t_0$,
  aplicaríamos análogamente el lema del primer instante para obtener un $\tau \in [t_0-a,t_0)$
  con $b < \| \varphi(\tau) - x_0 \|$ que cumpliría $\forall t \in (\tau,t_0]\colon \|\varphi(t)-x_0\| \leq b$.
  Pero entonces, de nuevo,
  \[\begin{aligned}
   b < \|\varphi(\tau)-x_0\| &=
   \norm{ -\int_{\tau}^{t_0} f(s,\varphi(s))\,ds } \leq
   \int^{t_0}_\tau \norm{ f(s,\varphi(s)) }\,ds \\&\leq
   M|\tau-t_0| \leq
   Ma \leq b,
   \end{aligned}\]
  y habríamos llegado igualmente a contradicción.
\end{proof}

Lo que probamos a continuación es que una condición de lipschitzianidad respecto
de la segunda variable para el campo que determina la ecuación diferencial nos
da una condición de lipschitzianidad sobre el operador de Volterra.

\begin{lemma}[]\label{lema-3-contr}
  Para $L \geq 0$ tal que
  \[
    \forall (t,x),(t,y) \in {\eR}_{a,b}\colon
    \| f(t,x) - f(t,y) \| \leq L \|x-y\|, 
  \]
  se tiene $\|V(\varphi) - V(\psi)\|_\infty \leq aL\|\varphi - \psi\|_\infty$ para cualesquiera $\varphi,\psi \in E_{a,b}$.
\end{lemma}
\begin{proof}
  Para $t \in [t_0-a,t_0+a]$, si suponemos $t > t_0$ tenemos
  \[\begin{aligned}
  \| V(\varphi)(t) - V(\psi)(t) \| &=
  \norm{ \int_{t_0}^t \left( f(s,\varphi(s)) - f(s,\psi(s)) \right)\,ds } \\&\leq
  \int_{t_0}^t \norm{  f(s,\varphi(s)) - f(s,\psi(s)) }\,ds \\&\leq
  \int_{t_0}^t L\norm{ \varphi(s)-\psi(s) }\,ds \\&\leq
  La\|\varphi-\psi\|_{\infty}.
  \end{aligned}\]
En el caso $t = t_0$, tendremos $\|V(\varphi)(t) - V(\psi)(t)\| = 0$;
y en el caso $t < t_0$, realizaremos un razonamiento análogo al anterior
pero ahora notando en el paso intermedio que
\[\begin{aligned}
  \norm{-\int_{t_0}^t  (f(s,\varphi(s)) - f(s,\psi(s))) \,ds} \leq
  \int_{t}^{t_0} \norm{f(s,\varphi(s)) - f(s,\psi(s))}\,ds.
\end{aligned}\]
Así, habremos acotado la diferencia en todos los casos con una cota independiente
de $t$ y por tanto hemos acotado la norma del supremo.
\end{proof}

Tomando ahora un $a$ suficientemente pequeño, podremos obtener una primera
aplicación del Teorema de Banach a nuestro caso particular.

\begin{proposition}\label{proposicion-1-contr}
Sean $M,L \geq 0$ tales que
\[\forall (t,x) \in R_{a,b}\colon \|f(t,x)\| \leq M,\]
y que
\[\forall (t,x),(t,y) \in {\eR}_{a,b}\colon \|f(t,x)-f(t,y)\| \leq L\|x-y\|.\]
Si $aM\leq b$ y $aL \leq 1$, entonces existe una única $\varphi\colon [t_0-a,t_0+a] \to \RR^d$
solución al problema de Volterra.
\end{proposition}
\begin{proof}
Aplicando el Corolario~\ref{corolario-1-volterra} a $M$, tenemos $V(E_{a,b}) \subset E_{a,b}$;
y aplicando el Lema~\ref{lema-3-contr} a $L$, obtenemos
\[
\| V(\varphi) - V(\psi) \|_{\infty} \leq aL\|\varphi-\psi\|_{\infty}.
\]
Ahora, como $aL < 1$, por el Teorema del punto fijo de Banach en $E_{a,b}$, que es completo,
$\exists! \varphi \in E_{a,b} \colon V(\varphi) = \varphi$
solución de Volterra. Si hubiera otra solución $\psi$ en mismo intervalo usaríamos
el Lema~\ref{lema-2-contr} para tener $\psi \in E_{a,b}$, y como el Teorema de Banach nos
da unicidad en ese espacio, $\varphi = \psi$.
\end{proof}

Nuestro siguiente objetivo será retirar la hipótesis $aL < 1$, para lo que
utilizaremos la norma de Bielecki.

\subsection{Norma de Bielecki}

La \textbf{norma de Bielecki}, para una constante \(R > 0\) y una norma fijada
en $\RR^d$, se define en el espacio \(E_{a,b}\) como
\[\|\varphi\|_B = \max_{t \in [t_0-a,t_0+a]} e^{-R|t-t_0|}\|\varphi(t)\|.\]

\begin{remark}[equivalencia con la norma del máximo]
  La norma de Bielecki es equivalente a la norma del máximo. Explícitamente,
  se tiene que
  \[
    \|\varphi\|_B = \max_{t \in [t_0-a,t_0+a]} e^{-R|t-t_0|} \|\varphi(t)\| \leq
     e^{-Ra} \|\varphi\|_\infty
   \]
  y que
  \[
   \|\varphi\|_\infty =  \max_{t \in [t_0-a,t_0+a]} e^{R|t-t_0|}e^{-R|t-t_0|} \|\varphi(t)\| \leq
   e^{Ra}\|\varphi\|_B.
  \]
\end{remark}

\begin{lemma}[versión del lema 3 con la norma de Bielecki]\label{lema-4-contr}
  Para \(L \geq 0\) tal que
  \[
    \forall (t,x),(t,y) \in {\eR}_{a,b} : \| f(t,x) - f(t,y) \| \leq L\|x-y\|,
  \]
  se tiene \(\|V(\varphi)-V(\psi)\|_B \leq \frac{L}{R} \|\varphi - \psi\|_B\)
  para cualesquiera $\varphi,\psi \in E_{a,b}$.
\end{lemma}
\begin{proof}
Nótese primero que $\|V(\varphi)(t_0) - V(\psi)(t_0)\| = 0$. Fijado un $t > t_0$
usamos la cota de $L$ y la definición de la norma de Bielecki para tener
\[\begin{aligned}
\norm{V(\varphi)(t) - V(\psi)(t)} &=
\norm{\int_{t_0}^t f(s,\varphi(s)) - f(s,\psi(s))\,ds} \leq
\int_{t_0}^t \norm{f(s,\varphi(s)) - f(s,\psi(s))}\,ds \\&\leq
L \int_{t_0}^t \norm{\varphi(s) - \psi(s)}\,ds =
L \int_{t_0}^t \norm{\varphi(s) - \psi(s)} e^{-R\abs{s-t_0}}e^{R\abs{s-t_0}}\, ds \\&\leq
L \norm{\varphi - \psi}_{B} \int_{t_0}^t e^{R\abs{s-t_0}}\,ds \leq
\frac{L}{R} \norm{\varphi - \psi}_{B} \left(e^{R\abs{t-t_0}} - 1\right) \\&\leq
\frac{L}{R} \norm{\varphi - \psi}_{B} e^{R\abs{t-t_0}}
\end{aligned}\]
usando en el último paso que $e^{R|t-t_0|} > 1$. Si fijamos un $t < t_0$
podemos repetir el mismo razonamiento simplemente notando al principio que
\[
\norm{- \int_{t_0}^t f(s,\varphi(s)) - f(s,\psi(s))\,ds} \leq
\int_{t}^{t_0} \norm{f(s,\varphi(s)) - f(s,\psi(s))}\,ds.
\]
Dividiendo y uniendo los tres casos, llegamos a que
$\|V(\varphi) - V(\psi)\|_{B} \leq \frac{L}{R} \norm{\varphi - \psi}_{B}$.
\end{proof}

Con esto obtenemos un enunciado similar al de nuestra primera proposición
pero habiendo eliminado la hipótesis que acotaba $aL$.

\begin{proposition}\label{proposicion-2-contr}
Sean $M,L \geq 0$ tales que
\[
\forall (t,x) \in R_{a,b}\colon \|f(t,x)\| \leq M,
\]
y que
\[
\forall (t,x),(t,y) \in {\eR}_{a,b}\colon \|f(t,x)-f(t,y)\| \leq L\|x-y\|.
\]
Si $aM\leq b$, entonces existe una única $\varphi\colon [t_0-a,t_0+a] \to \RR^d$
solución al problema de Volterra.
\end{proposition}
\begin{proof}
  Si tomamos $R > L$, podemos aplicar el Lema~\ref{lema-4-contr} para
  obtener que $V$ es contractiva en el espacio métrico $E_{a,b}$ con la
  norma de Bielecki. Repitiendo el razonamiento de la Proposición~\ref{proposicion-1-contr},
  obtenemos que $V(E_{a,b}) \subset E_{a,b}$ y que existirá un punto
  fijo de $V$ por el Teorema de Banach en $E_{a,b}$ que sigue siendo
  completo bajo la norma de Bielecki, equivalente a la norma del supremo.
\end{proof}

\section{Teorema de Picard-Lindelöf}

\subsection{Teorema de Picard-Lindelöf}

\begin{theorem}[Picard-Lindelöf local]\label{picard-lindelof-local}
  Si \(f\) es localmente lipschitziana respecto de \(x\) en un entorno ${\cal U}$ de \((t_0,x_0)\),
  entonces existe solución al problema de valores iniciales y es
  localmente única.
\end{theorem}
\begin{proof}
  Por hipótesis sabemos que existe una constante de Lipschitz $L \in \RR^+$
  tal que
  \[ \forall (t,x),(t,y) \in {\cal U}\colon \| f(t,x) - f(t,y) \| \leq L\|x-y\|. \]
  Usando que la familia de entornos ${\eR}_{a,b}$ definida en la Sección~\ref{base-entornos}
  forma una base de entornos, sabemos que existen $\overline{a},b \in \RR^+$ tales que ${\eR}_{\overline{a},b}(t_0,x_0) \subseteq {\cal U}$.
  Ahora, $\restr{f}{{\eR}_{\overline{a},b}}$ será globalmente lipschitziana respecto de $x$;
  siendo continua, podremos aplicar el Teorema de Weierstrass para obtener la cota
  \[M = \max_{(t,x) \in {\eR}_{\overline{a},b}} \|f(t,x)\|.\]
  Definimos ahora $a = \min\left\{ \overline{a}, b/M, 1/L \right\}$ y, sabiendo que ${\eR}_{a,b} \subseteq {\eR}_{\overline{a},b}$, podemos aplicar
  la Proposición~\ref{proposicion-2-contr} para obtener que existe una solución
  localmente única $\varphi \colon [t_0-a,t_0+a] \to \RR^d$ a la ecuación de Volterra, que en el
  abierto $\varphi \colon I = (t_0-a,t_0+a) \to \RR^d$ será solución del PVI\@. Nótese sin embargo que
  todavía no hemos probado la unicidad local para el PVI\@.
  
  Sea ahora $\psi \colon J \to \RR^d$ otra solución en un intervalo abierto conteniendo
  a $t_0$ (si estuviera definida en un intervalo cerrado con extremo en $t_0$ podríamos
  concatenarla con $\varphi$ y seguir probando unicidad); por continuidad existe un $\delta' < a$
  tal que para $|t - t_0| < \delta'$ se tiene $\|\psi(t) - x_0\| \leq b$. Como ambos
  intervalos son abiertos y $t_0 \in I \cap J$, existirá
  un entorno de $t_0$ contenido en la intersección y podremos tomar un
  $\delta < \delta'$ suficientemente pequeño para asegurar que $(t_0-\delta, t_0+\delta) \subset I \cap J$.
  Tendríamos así dos soluciones al problema de Volterra $\psi,\varphi \in E_{\delta,b}(t_0,x_0)$,
  donde además sabemos que $\delta M < aM < b$, luego, de nuevo por la
  Proposición~\ref{proposicion-2-contr}, coincidirán en $[t_0-\delta,t_0+\delta]$.
\end{proof}


\begin{theorem}[Picard-Lindelöf global]\label{picard-lindelof-global}
  Si \(f\) es localmente lipschitziana respecto de \(x\), entonces existe solución
  al problema de valores iniciales y es la única solución maximal.
\end{theorem}
\begin{proof}
  Cualquier $(t_0,x_0)$ tiene un entorno en el dominio donde podemos aplicar
  la versión local del Teorema~\ref{picard-lindelof-local}. Finalmente, sabemos
  que la propiedad de unicidad local bajo cualesquiera condiciones iniciales
  implica la unicidad global bajo cualquier condición inicial.
\end{proof}



\subsection{Iterantes de Picard}

La segunda consecuencia que obtenemos del Teorema de Banach en estas
condiciones es la posibilidad de aproximar la solución del PVI
mediante la aplicación sucesiva del operador de Volterra en los casos
en los que hemos comprobado que es contractivo.

\begin{theorem}[Iterantes de Picard]
  Supongamos que tenemos constantes \(a,b,M,L > 0\) y un dominio \(D \subset \RR \times \RR^d\)
  cumpliendo
  \begin{itemize}
    \item \(a \leq b/M\),
    \item \({\eR}_{a,b} \subset D\),
    \item \(\forall (t,x) \in {\eR}_{a,b}\colon \|f(t,x)\| \leq M\),
    \item \(\forall (t,x),(t,y) \in {\eR}_{a,b} \colon \|f(t,x)-f(t,y)\| \leq L \|x-y\|\).
  \end{itemize}
  Entonces la sucesión de funciones definida recursivamente mediante aplicaciones
  sucesivas del operador integral de Volterra
  \begin{itemize}
    \item \(\phi_0(t) = x_0\),
    \item \(\displaystyle \phi_{k+1}(t) = x_0 + \int_{t_0}^t f(s,\phi_k(s))\,ds\),
  \end{itemize}
  está bien definida y converge uniformemente hacia la única solución
  de la ecuación integral de Volterra. Esta sucesión, \(\left\{ \phi_n \right\}\), se conoce
  como sucesión de \textbf{iterantes de Picard}.
\end{theorem}

\begin{remark}[ejemplo de iterantes de Picard]
  Consideremos $f(t,x) = 2tx$ con $x(0) = 1$ en el dominio $D = \RR\times\RR$, y tomemos
  las constantes $a = 1/2$, $b=1$, $M=2$, $L=1$, para las que
  podemos comprobar que se está en condiciones de aplicar los
  iterantes de Picard. Obtenemos
  \begin{itemize}
  \item $\phi_0(t) = 1$,
  \item $\displaystyle\phi_1(t) = 1 + \int_{0}^t 2s\,ds = 1 + t^2$,
  \item $\displaystyle\phi_2(t) = 1 + \int_{0}^t 2s(1 + s^2) \,ds = 1 + t^2 + \frac{1}{2}s^4$,
  \item $\displaystyle\phi_3(t) = 1 + \int_{0}^t 2s(1 + s^2 + \frac{1}{2}s^4) \,ds = 1 + t^2 + \frac{1}{2}s^4 + \frac{1}{6} s^6$,
  \end{itemize}
  que podemos ver que se van aproximando a la serie de Taylor de
  la exponencial evaluada en $t^2$, como sería esperable tras
  resolver la ecuación por variables separadas.
\end{remark}

\subsection{Contraejemplo de Müller}

El \textbf{contraejemplo de Müller} (1927) considera el PVI
\[\left\{\begin{array}{l}
  x' = f(t,x) \\
  x(0) = 0
\end{array}\right.\]
para \(f \colon \RR \times \RR \to \RR\) una función definida a trozos como
\[ f(t,x) = \left\{
    \begin{array}{ll}
      0 & \mbox{ si } t \leq 0, \\
      2t & \mbox{ si } t > 0 \mbox{ y } x < 0, \\
      2t - \frac{4x}{t} & \mbox{ si } t > 0 \mbox{ y } 0 \leq x < t^2, \\
      -2t & \mbox{ si } t > 0 \mbox{ y } x > t^2. \\
    \end{array}\right.\]
Este es un PVI con una solución única a la que no se aproximará
la sucesión de iterantes de Picard, ni siquiera en ninguna de sus
parciales.


\begin{figure}
\centering
\begin{tikzpicture}[
  declare function={
    func(\t,\x)= (\t>0) * (
      (\x<0) *               (2*\t) +
      and(\x>=0, \x<\t*\t) * (2*\t - 4*\x/\t) +
      (\x>\t*\t) *           (-2*\t));
  }
  ]
\begin{axis}[
  title={Solución del contraejemplo de Müller},
  domain=-2:4,
  view={0}{90},
  axis background/.style={fill=white},
  ]
  \addplot[blue, dashed, samples=101, domain=-2:2.05]{(x>0) * x^2};
  \addplot[blue, dashed, samples=101, domain=-2:4]{0};  
  \addplot3[blue,
     quiver={
      u={1  / sqrt( func(x,y)^2 + 1 )},
      v={func(x,y) / sqrt( func(x,y)^2 + 1 )},
      scale arrows=0.3,
     },
     -stealth,samples=15]
     {exp(0-x^2-y^2)*x};
  \addplot[samples=101, domain=-2:3.6]{(x>0) * x^2/3};     
\end{axis}
\end{tikzpicture}
\end{figure}

Podemos comprobar que tiene una única solución aplicando el Teorema
de Unicidad de Peano. En efecto, la unicidad en el futuro se obtiene
del hecho de que para cualquier $t>0$, la función
\[ f(t,x) = \left\{
    \begin{array}{ll}
      2t & \mbox{ si } x < 0, \\
      2t - \frac{4x}{t} & \mbox{ si } 0 \leq x < t^2, \\
      -2t & \mbox{ si } x > t^2, \\
    \end{array}\right.\]
está formada por tres trozos de recta y es decreciente en $x$. Del
mismo modo, la unicidad en el pasado se obtiene del hecho de que
para $t < 0$, se tiene $f(t,x) = 0$ y por tanto es creciente. De
la conjunción de las propiedades de unicidad en el futuro y en el
pasado obtenemos la unicidad global de la solución, si existiera.

Además, podemos calcular explícitamente la solución. Si buscamos
soluciones a la ecuación lineal
\[
x' = 2t - \frac{4x}{t};
\]
obtendremos posibles soluciones de la forma $x = t^2/3 + A/t^4$. En
nuestro caso, tendremos que tomar $A = 0$ para que pueda cumplirse
la condición de valores iniciales y que esté bien definida en $0$.
Concatenaremos con una solución constantemente $0$ antes de la
condición inicial para comprobar que, efectivamente,
\[
  \varphi(t) = \left\{\begin{array}{ll}
      0 &\mbox{ si } t < 0 \\
      t^2/3 &\mbox{ si } t \geq 0   
  \end{array}\right.
\]
es continua y derivable y es solución de nuestro problema de
valores iniciales, ya que para $t > 0$ se tiene $f(t,t^2/3) = 2t - 4/3t = 2/3t = \varphi'(t)$,
donde usamos crucialmente que $t^2/3 < t^2$.

Sin embargo, si intentamos aproximarla por iterantes de Picard
obtendríamos,
\begin{itemize}
\item $\phi_0(t) = 0$,
\item $\displaystyle\phi_1(t) = \int_0^t f(s,0)\,ds = \int_0^t 2s\,ds = t^2$,
\item $\displaystyle\phi_2(t) = \int_0^t f(s,s^2)\,ds = \int_0^t -2s\,ds = -t^2$,
\item $\displaystyle\phi_3(t) = \int_0^t f(s,-s^2)\,ds = \int_0^t 2s\,ds = t^2$, $\dots$
\end{itemize}
y sucesivamente iteraríamos entre $t^2$ y $-t^2$, que no se acercan
a la solución ni siquiera en ninguna de sus parciales.


\section{Teorema de Cauchy-Peano}

El Teorema de Cauchy-Peano nos asegura la existencia de solución a
un PVI exigiendo simplemente la continuidad.

En esta sección usaremos el Teorema de Stone-Weierstrass, que permite
aproximar uniformemente funciones continuas en un compacto por una familia
de funciones a la que sólo le pedimos algunas propiedades algebraicas.
Aunque sólo necesitaremos la aproximación mediante familias de polinomios
en varias variables, probaremos la versión más general, que abstrae las
consideraciones particulares sobre los polinomios a la estructura de
\textit{álgebra} de funciones.

\subsection{Teorema de Stone-Weierstrass}

Una familia de funciones reales ${\cal A}$ sobre un conjunto $X$ se dice
\textbf{álgebra} si es cerrada para la suma y producto de funciones y
para el producto por escalares. 

Si consideramos el conjunto de funciones $X \to \mathbb{R}$ como un
espacio métrico con la norma del supremo, la convergencia es la
convergencia uniforme. En el caso en el que todas las funciones
de ${\cal A}$ están acotadas, podemos comprobar que la clausura bajo esta
norma de ${\cal A}$ es de nuevo un \textbf{álgebra}; de hecho, tenemos que para
$\lambda \in \mathbb{R}$ y $p_n,q_n \in {\cal A}$ con $p_n \overset{c.u.}\longrightarrow p$ y $q_n \overset{c.u.}\longrightarrow q$,

\begin{itemize}
\item $p_n + q_n \overset{c.u.}\longrightarrow p + q$, ya que
   \[
   \norm{(p_n + q_n) - (p + q)}_{\infty} \leq
   \norm{p_n - p}_{\infty} + \norm{q_n - q}_{\infty} \longrightarrow 0;
   \]
\item $p_nq_n \overset{c.u.}\longrightarrow pq$ usando que todas ellas estarán acotadas y
      que $\norm{p}_{\infty} \leq \sup \left\{ \norm{p_n}_{\infty} \right\}$, que existe por la convergencia,
   \[
   \norm{p_nq_n - pq}_{\infty} \leq
   \norm{p_n - p}_{\infty}\sup\set{\norm{q_n}_{\infty}} + 
   \sup \left\{ \|p_n\|_{\infty} \right\} \norm{q_n-q}_{\infty} \longrightarrow 0;
   \]
\item y $\lambda p_n \overset{c.u.}\longrightarrow \lambda p$, ya que $\norm{\lambda p_n}_{\infty} \leq \lambda\norm{p_{\infty}} \longrightarrow 0$.
\end{itemize}

Decimos que el álgebra \textbf{separa puntos} si para cada par $x,y \in X$
existe $p \in {\cal A}$ tal que $p(x) \neq p(y)$. Decimos que no se \textbf{desvanece}
en ningún punto si para cada $x \in X$ hay una $p \in {\cal A}$ tal que $p(x) \neq 0$.
Cuando un álgebra separa puntos y no se devanece, dados dos reales
$\lambda,\mu \in \mathbb{R}$ y dos puntos $x,y \in X$ existe una $p \in {\cal A}$ que separa
los puntos y dos $q_x,q_y \in {\cal A}$, con $q_x(x) \neq 0$ y $q_y(y) \neq 0$; entonces
podemos definir
\[
a = pq_x - p(y)q_x, \quad
b = pq_y - p(x)q_y, 
\]

que cumplen $a(x) \neq 0$ y $b(y) \neq 0$, $b(x) = a(y) = 0$,; así, existe
siempre una función $r$ que vale exactamente $\lambda$ en $x$ y $\mu$ en $y$,
definida como
\[
r = \frac{\lambda b}{b(x)} + \frac{\mu a}{a(y)}.
\]


\begin{theorem}[de Stone-Weierstrass]\label{stone}
Sea ${\cal A}$ un álgebra de funciones continuas sobre un espacio
compacto $K$ que separa puntos y no se desvanece en ninguno.
La clausura bajo la norma del supremo es entonces ${\cal C}(K,\mathbb{R})$,
el conjunto de todas las funciones continuas reales sobre $K$.
\end{theorem}
\begin{proof}
  Nuestro \textit{primer paso} será aproximar uniformemente la raíz cuadrada
por una sucesión de polinomios. La serie de Taylor de la raíz
cuadrada en $t = 1$,
\[\sum_{n\geq 0} \lambda_n(t-1)^n =
\sum_{n\geq 0} \frac{(t - 1)^n}{n!} \prod_{k=1}^n \left( \frac{3}{2} - k \right),
\]
tiene coeficientes $\lambda_n$ tales que
\[
\abs{\frac{\lambda_{n+1}}{\lambda_n}} = 
\abs{\frac{1/2 - n}{n+1}} \to 1,
\qquad
n \left( 1 - \abs{\frac{\lambda_{n+1}}{\lambda_n}} \right) \to 3/2 > 1;
\]
por lo que sabemos por criterio de Raabe que $\sum_{n \geq 1} \lambda_n$ es absolutamente
convergente y la suma converge para $t \in [0,1]$. Podemos aproximar $\psi(t) = \sum_{n=0}^{\infty}\lambda_n(t-1)^n$,
en $t \in [0,1]$ uniformemente por polinomios usando que
\[
\abs{\psi(t) - \sum_{n=0}^{m} \lambda_n(t-1)^n} \leq
\sum_{n=m+1}^{\infty} \abs{\lambda_n},
\quad\mbox{ para } t \in [0,1].
\]
Si ahora consideramos sus derivadas, usando convergencia uniforme para
intercambiar derivada y suma infinita, tenemos la ecuación diferencial
\[\begin{aligned}
\psi'(t) &=
\sum_{n=0}^\infty n\lambda_n(t-1)^{n-1} =
\sum_{n=1}^\infty \left(\frac{1}{2} - (n-1)\right)\lambda_{n-1}(t-1)^{n-1} \\&=
\frac{1}{2}\psi(t) - (t-1)\psi'(t),
\end{aligned}\]
con $\psi(1) = 1$, que se resuelve como $\psi(t) = \sqrt{t}$ en $(0,1]$ y
se extiende únicamente por continuidad a $[0,1]$.

Nuestro \textit{segundo paso} utiliza lo anterior para dar estructura de
retículo a $\overline{{\cal A}}$.
Dada $p \in \overline{{\cal A}}$, veamos que $\abs{p} \in {\overline{\cal A}}$. Al ser $K$ compacto, $p$ está acotado.
Dado un $\varepsilon > 0$, como $p(t)/\norm{p}_{\infty} \leq 1$, usamos la aproximación por polinomios
de la raíz cuadrada para existir algún $n$ tal que
\[
\abs{\sqrt{\left(\frac{p(x)}{\norm{p}_{\infty}}\right)^2} -
\sum_{i=1}^n \lambda_i \left(\frac{p(x)}{\norm{p}_{\infty}}-1\right)^i} < \varepsilon;
\quad
\mbox{ para } x \in K.
\]
Y desde aquí deducimos que $\abs{p}/\norm{p}_{\infty} \in \overline{{\cal A}}$ por
poder aproximarse uniformemente por sumas y productos de funciones en el álgebra,
luego $\abs{p} \in \overline{\cal A}$.

El ser cerrada para el valor absoluto hace que la clausura
sea un retículo con el orden parcial inducido punto a punto
desde los reales. En otras palabras, dados $p,q\in \overline{{\cal A}}$,
\[
\min(p,q) = \frac{p + q}{2} - \frac{\abs{p-q}}{2},\quad\text{y}\quad
\max(p,q) = \frac{p + q}{2} + \frac{\abs{p-q}}{2},
\]
pertenecen a $\overline{\cal A}$.

Nuestro \textit{tercer paso} toma $f \in {\cal C}(K,\mathbb{R})$ y un $\varepsilon > 0$.
Por ser ${\cal A}$ separable y no desvanecerse, tenemos para cada $z,y \in K$ una
función $s_z \in {\cal A}$ tal que $s_{z}(y) = f(y)$ y además, $s_{z}(z) = f(z)$.
Por continuidad de $s_z$, existe un abierto $z \in J_z$ tal que
\[
\forall x \in J_z\colon\quad s_z(x) > f(x) - \varepsilon.
\]
Como $K = \bigcup_{z \in K} J_z$ es compacto, existen $z_1,\dots,z_n$ tales que
$K = J_{z_1} \cup \dots \cup J_{z_n}$; y podemos tomar $r_y = \max(s_{z_1},\dots,s_{z_n})$.
De nuevo, por continuidad de $r_y$, existe un abierto $y \in H_y$
tal que
\[
\forall x \in H_y\colon\quad r_y(x) < f(x) + \varepsilon.
\]
Usando compacidad de $K$, existen $y_1,\dots,y_n$ con $K = H_{y_1} \cup \dots \cup H_{y_m}$;
y podemos tomar $h~=~\min(r_{y_1},\dots,r_{y_m}) \in \overline{{\cal A}}$; que cumple
que $f(x) - \varepsilon < h(x) < f(x) + \varepsilon$, y por tanto $\norm{f - h}_{\infty} < \varepsilon$.

\end{proof}

\subsection{Teorema de Cauchy-Peano}

\begin{lemma}[de aproximación por localmente lipschitzianas]\label{aprox-ll}
  Sea $D \subseteq \RR \times \RR^d$ abierto con $(t_0,x_0) \in D$ y $f \colon D \to \RR$ una función continua.
  Para \(M \geq 0\) cumpliendo que \(\forall (t,x)\in {\eR}_{a,b}(t_0,x_0) : \|f(t,x)\| \leq M\), existe
  una sucesión de funciones $f_n \colon D \to \mathbb{R}^d$ localmente lipschitzianas
  tales que
  \[
    \forall n \in \ZZ^+\colon
    \forall (t,x) \in {\eR}_{a,b}\colon
    \norm{f_n(t,x)} \leq M,
  \]
  cumpliendo que $f_n$ converge uniformemente a $f$ en ${\eR}_{a,b}$.
\end{lemma}
\begin{proof}
  Aplicamos el Teorema de Stone-Weierstrass (Teorema~\ref{stone}) en
  el compacto ${\eR}_{a,b}$, al álgebra de los polinomios en varias variables
  \[
    {\cal A} =
    \left\{ \sum_{(i_1,\dots,i_n) \in \NN^{d+1}}
      \left(\lambda_{i_1,\dots,i_n} \prod_{j=1}^n x_j^{i_j}\right)
      \midd
      \lambda_{i_1,\dots,i_n} \in \RR
    \right\},
  \]
  que es cerrada para la suma, el producto por escalares y el producto
  de funciones y que además contiene funciones que separan puntos (con las
  lineales basta) y funciones que no se desvanecen en cada uno de ellos.

  Trabajando en ${\eR}_{a,b}$ compacto con la norma del supremo, podemos aproximar
  uniformemente la función $f$ por polinomios tanto como queramos, así que
  tomamos $\set{p_n}$ tales que $\norm{p_n - f}_\infty < 1/n$. Nótese que cumplen $\norm{p_n}_\infty \leq M + 1/n$,
  por lo que podemos definir la siguiente sucesión de funciones, que son
  localmente lipschitzianas por ser polinomios,
  \[
    f_n = \frac{M}{M+1/n}p_n;\quad\mbox{ cumpliendo }\quad
    \norm{f_n}_\infty \leq M.
  \]
  Y tenemos finalmente la siguiente aproximación uniforme,
  \[
    \norm{\frac{M}{M + 1/n} p_n - f}_\infty \leq
    \norm{\frac{1/n}{M + 1/n}p_n}_\infty + \norm{p_n - f}_\infty \leq
    \frac{1}{n} + \frac{nM}{nM+1} \to 0.\qedhere
  \]
\end{proof}

\begin{proposition}\label{propcauchypeano}
  Sea $D \subseteq \RR \times \RR^d$ abierto con $(t_0,x_0) \in D$ y $f \colon D \to \RR$ una función continua.
  Para \(M \geq 0\) cumpliendo que \(\forall (t,x)\in {\eR}_{a,b}(t_0,x_0) : \|f(t,x)\| \leq M\) y
  que $aM \leq b$, existe una función $\varphi \colon [t_0-a,t_0+a] \to \RR^d$ que es solución
  del problema de valores iniciales asociado.
\end{proposition}
\begin{proof}
  Por el Lema~\ref{aprox-ll}, tenemos una aproximación en ${\eR}_{a,b}$ por localmente
  lipschitzianas $f_n \overset{c.u.}\longrightarrow f$, cumpliendo $f_n(x) \leq M$ en ${\eR}_{a,b}$, donde
  son además lipschitzianas por compacidad. Si consideramos la familia de ecuaciones
  de Volterra asociadas,
  \[
    x(t) = x_0 + \int_{t_0}^t f_n(s,x(s))\,ds,
  \]
  sabemos por la Proposición~\ref{proposicion-2-contr} que existe una sucesión
  de soluciones $\varphi_n \colon [t_0-a,t_0+a]\to\RR^d$ a cada uno de los problemas de Volterra,
  y por el Lema~\ref{lema-2-contr}, sabemos que $\varphi_n \in E_{a,b}$.
  Ahora, las derivadas $\varphi'_n(t) = f(t,\varphi(s)) \geq M$ están uniformemente acotadas por $M$,
  por lo que la sucesión es equicontinua, y además, las $\varphi_n$ están uniformemente acotadas
  por
  \[
    \varphi_n(t) = x_0 + \int_{t_0}^t f_n(s,\varphi_n(s))\,ds \leq x_0 + aM.
  \]
  Aplicando ahora el \textit{Teorema de Ascoli-Arzelá}, tenemos una una parcial uniformemente
  convergente $\varphi_{\sigma(n)} \overset{c.u.}\longrightarrow \varphi$. Podemos tomar límites y usar el \textit{Teorema de la convergencia
  dominada de Lebesgue} sabiendo $f_n(s,\varphi_n(s)) \leq M$ para tener
  \[\begin{aligned}
    \varphi(t) &=
    \lim_{n \to \infty} \left(x_0 + \int_{t_0}^t f_n(s,\varphi_n(s))\,ds\right) \\&=
    x_0 + \int_{t_0}^t \lim_{n \to \infty} f_n(s,\varphi_n(s))\,ds =
    x_0 + \int_{t_0}^t f(s,\varphi(s))\,ds.
  \end{aligned}\]
  Donde usamos que $f_{\sigma(n)}(s,\varphi_{\sigma(n)}(s)) \to f(s,\varphi(s))$ por tenerse $f_n \overset{c.u.}\longrightarrow f$. La $\varphi$,
  que es continua por ser límite uniforme de continuas, es una solución.
\end{proof}
\begin{theorem}[de Cauchy-Peano]
  Para $D = \mathring{D} \subset \mathbb{R} \times \mathbb{R}^d$ y $f \colon D \to \mathbb{R}^{d}$ continua con $(t_0,x_0) \in D$, el PVI
 \[\left\{\begin{array}{c}
 x' = f(t,x), \\
 x(t_0) = x_0,
          \end{array}\right.\]
  tiene solución.
\end{theorem}
\begin{proof}
  Siendo $D$ abierto, existen $\overline{a},b \in \mathbb{R}^{+}$ tales que
  $\eR_{a,b} \subseteq D$. En este compacto, por \textit{Teorema de Weierstrass},
  tenemos que $\exists M\colon \forall (t,x) \in \eR_{a,b}\colon \norm{f(t,x)} \leq M$.
  Podemos ahora tomar $a = \min \set{b/M,\overline{a}}$ para tener las condiciones
  de la Proposición~\ref{propcauchypeano} en $\eR_{a,b}$. Esto nos dará una solución al
  Problema de Volterra en el intervalo $[t_0-a,t_0+a]$ que será solución al PVI
  en $(t_0-a,t_0+a)$.
\end{proof}


\bibliography{Bibliography}
\bibliographystyle{alpha}

La demostración del Teorema de Stone-Weierstrass la tomamos
de~\cite{rudin64}; pero en lugar de aproximar la raíz cuadrada usando el
Teorema de aproximación de Weierstrass, como se hace allí, 
hemos tomado la idea de aproximar la raíz cuadrada usando su serie de
Taylor y resolviendo la ecuación diferencial de~\cite{queen}.

\begin{appendices}
\section{Implementación de iterantes de Picard}
A continuación presentamos una implementación del proceso de cálculo
de iterantes de Picard.

\begin{verbatim}
#!/usr/bin/env sage
# -*- coding: utf-8 -*-

# Iterantes de Picard
# SageMath version 8.1
# Autor: Mario Román
def picard(f,t0,x0,n):
    if n == 0: return (lambda s: x0)
    phi_ant = picard(f,t0,x0,n-1)
    return (lambda s: x0 + integral(f(s,phi_ant(s)),s,t0,t))

var('t')
print("Para $f(t,x) = x$, se tiene $\displaystyle\phi_5(t)={}$.\\\\"
      .format(latex(picard(lambda t,x: x, 0,1, 5)(t))))
print("Para $f(t,x) = -x$, se tiene $\displaystyle\phi_5(t)={}$.\\\\"
      .format(latex(picard(lambda t,x: -x, 0,1, 5)(t))))
print("Para $f(t,x) = 2tx$, se tiene $\displaystyle\phi_5(t)={}$.\\\\"      
      .format(latex(picard(lambda t,x: 2*t*x, 0,1, 5)(t))))
\end{verbatim}

De este código obtenemos en \LaTeX\ la salida siguiente.

Para $f(t,x) = x$, se tiene $\displaystyle\phi_5(t)=\frac{1}{120} \, t^{5} + \frac{1}{24} \, t^{4} + \frac{1}{6} \, t^{3} + \frac{1}{2} \, t^{2} + t + 1$.\\
Para $f(t,x) = -x$, se tiene $\displaystyle\phi_5(t)=-\frac{1}{120} \, t^{5} + \frac{1}{24} \, t^{4} - \frac{1}{6} \, t^{3} + \frac{1}{2} \, t^{2} - t + 1$.\\
Para $f(t,x) = 2tx$, se tiene $\displaystyle\phi_5(t)=\frac{1}{120} \, t^{10} + \frac{1}{24} \, t^{8} + \frac{1}{6} \, t^{6} + \frac{1}{2} \, t^{4} + t^{2} + 1$.\\

Asumiendo en todos los casos $x(0) = 1$.
\end{appendices}

\end{document}
